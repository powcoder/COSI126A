\documentclass[12pt]{article}

\usepackage[utf8]{inputenc}
\usepackage{latexsym,amsfonts,amssymb,amsthm,amsmath}
\usepackage{enumitem}
\usepackage{tabu}
\usepackage{graphicx}
\usepackage [english]{babel}
\usepackage [autostyle, english = american]{csquotes}
\MakeOuterQuote{"}
\graphicspath{ {.} }
\setlength{\parindent}{0in}
\setlength{\oddsidemargin}{0in}
\setlength{\textwidth}{6.5in}
\setlength{\textheight}{8.8in}
\setlength{\topmargin}{0in}
\setlength{\headheight}{18pt}


\title{COSI 126A: Homework 3\\ \large Due by Nov. 19th}
\date{}
\begin{document}

\maketitle

\vspace{0.5in}

\section*{Section I: Association Problems (50 points)}
\vspace{0.5in}

\subsection*{Problem 1 (10 points)}

\begin{center}
	\begin{tabu} to 0.75\textwidth { | X[c] | X[c] | }
	 \hline
	Transaction ID & Items Bought \\
	\hline
	1 & \{Milk, Beer, Diapers\}  \\
	\hline
	2 & \{Bread, Butter, Milk\} \\
	\hline
	3 & \{Milk, Diapers, Cookies\} \\
	\hline
	4 & \{Bread, Butter, Cookies\} \\
	\hline
	5 & \{Beer, Cookies, Diapers\} \\
	\hline
	6 & \{Milk, Diapers, Bread, Butter\} \\
	\hline
	7 & \{Bread, Butter, Diapers\} \\
	\hline
	8 & \{Beer, Diapers\} \\
	\hline
	9 & \{Milk, Diapers, Bread, Butter\} \\
	\hline
	10 & \{Beer, Cookies\} \\
	\hline
	\end{tabu}
\end{center}
\vspace{5mm}

Consider the market basket transactions shown above.

\begin{enumerate}[label=(\alph*)]
	\item What is the maximum number of association rules that can be extracted from this data (including rules that have zero support)?
	\item What is the maximum size of frequent itemsets that can be extracted (assuming $minsup > 0$)?
  \item Write an expression for the maximum number of size-3 itemsets that can be derived from this data set.
  \item Find an itemset (of size 2 or larger) that has the largest support.
  \item Find a pair of items, a and b, such that the rules $\{a\} \longrightarrow \{b\}$ and $\{b\} \longrightarrow \{a\}$ have the same confidence. 
\end{enumerate}

\subsection*{Problem 2 (10 points)}

Consider the following set of frequent 3-itemsets:
\begin{center}
$\{1, 2, 3\} , \{ 1 , 2 , 4 \} , \{ 1 , 2 , 5 \} , \{ 1 , 3 , 4 \} , \{ 1 ,3 , 5 \} , \{ 2 , 3 , 4 \} , \{ 2 , 3 , 5 \} , \{ 3 , 4 , 5 \}$.
\end{center}

Assume that there are only five items in the data set.

\begin{enumerate}[label=(\alph*)]
  \item List all candidate 4-itemsets obtained by a candidate generation procedure using the $F_{k-1} \times F_{1}$ merging strategy.
  \item List all candidate 4-itemsets obtained by the candidate generation procedure in $Apriori$.
  \item List all candidate 4-itemsets that survive the candidate pruning step of the $Apriori$ algorithm.
\end{enumerate}

\subsection*{Problem 3 (10 points)}

The original association rule mining formulation uses the support and confidence measures to prune uninteresting rules.

\begin{enumerate}[label=(\alph*)]
  \item Draw a contingency table for each of the following rules using the transactions shown in the table below.

\begin{center}
	\begin{tabu} to 0.5\textwidth { | X[c] | X[c] | }
	 \hline
	Transaction ID & Items Bought \\
	\hline
	1 & $\{a, b, c, e\}$  \\
	\hline
	2 & $\{b, c, d\}$ \\
	\hline
	3 & $\{a, b, d, e\}$ \\
	\hline
	4 & $\{a, c, d, e\}$ \\
	\hline
	5 & $\{b, c, d, e\}$ \\
	\hline
	6 & $\{b, d, e\}$ \\
	\hline
	7 & $\{d, e\}$ \\
	\hline
	8 & $\{a, b, c\}$ \\
	\hline
	9 & $\{a, d, e\}$ \\
	\hline
	10 & $\{b, d\}$ \\
	\hline
	\end{tabu}
\end{center}

Rules: $\{b\} \longrightarrow \{c\}$, $\{a\} \longrightarrow \{d\}$, $\{b\} \longrightarrow \{d\}$, $\{e\} \longrightarrow \{c\}$, $\{c\} \longrightarrow \{a\}$.

  \item Use the contingency tables in part (a) to compute and rank the rules in decreasing order according to the following measures.
\begin{enumerate}[label=(\alph*)]
  \item Support.
  \item Confidence.
  \item Interest$(X \longrightarrow Y) = \frac{P(X,Y)}{P(X)}P(Y)$.
  \item IS$(X \longrightarrow Y) = \frac{P(X,Y)}{\sqrt{P(X)P(Y)}}$.
  \item Klosgen$(X \longrightarrow Y) = \sqrt{P(X,Y)} \times (P(Y \mid X)-P(Y))$, where $P(Y \mid X) = \frac{P(X,Y)}{P(X)}$.
  \item Odds ratio$(X \longrightarrow Y) = \frac{P(X,Y)P(\overline{X},\overline{Y})}{P(X,\overline{Y})P(\overline{X},Y)}$.
\end{enumerate}
\end{enumerate}

\subsection*{Problem 4 (10 points)}
Given the rankings you had obtained in Exercise 12, compute the correlation between the rankings of confidence and the other five measures. Which measure is most highly correlated with confidence? Which measure is least correlated with confidence?

\subsection*{Problem 5 (10 points)}
Suppose we have market basket data consisting of 100 transactions and 20 items. If the support for item $a$ is 22\%, the support for item $b$ is 91\% and the support for itemset $\{a, b\}$ is 17\%. Let the support and confidence thresholds be 10\% and 60\%, respectively.

\begin{enumerate}[label=(\alph*)]
  \item Compute the confidence of the association rule $\{a\} \longrightarrow \{b\}$. Is the rule interesting according to the confidence measure?
  \item Compute the interest measure for the association pattern $\{a, b\}$. Describe the nature of the relationship between item $a$ and item $b$ in terms of the interest measure.
  \item What conclusions can you draw from the results of parts (a) and (b)?
\end{enumerate}


\pagebreak

\section*{Section II: Programming (50 points)}

Provided is a transformation of the Online Retail dataset. The transformed dataset contains 541,909 transactions and 2603 items. The meaning of each item is given in the file \textit{OnlineRetailAttributes.xlsx}.

\vspace{5mm}

You must find the 100 item pairs with the most support, as well as the confidence of the implied association rules.

\vspace{5mm}

Please consult \textit{HW3\_Association\_Rules.ipynb} for more information. Implement your assignment in this file and include it with your submission. As usual, the problem set should be submitted as a PDF, with LaTeX strongly encouraged.

\end{document}
